\chapter{Miscellaneous}\label{chapter-misc} 

\section{Integrability criterion}\label{sec:criterion}
The integrability criterion \citep{integrability-criterion} is a sufficient condition for a vector field to be a gradient field as well. It states that for some open, simple connected set $U$, a continuously differentiable function $F\,:\,U\rightarrow R^L$ defines a gradient field if and only if
\begin{equation}
\frac{\partial F_j(\textbf{x})}{\partial x_i} = \frac{\partial F_i(\textbf{x})}{\partial x_j}, \quad \forall i,j = 1...L
\end{equation}
In other words, integrability follows from the symmetry of the partial derivatives.

\section{Gradient with respect to gamma}\label{sec:log-gradient}
The gradient of the loss with respect to the coupling parameter $\gamma_{ij}$ is computed with the chain rule:
\begin{equation}
\frac{\partial \tilde{\mathcal{L}}}{\partial \gamma_{ij}} = \frac{\partial \tilde{\mathcal{L}}}{\partial E_i} \cdot \frac{\partial E_i}{\partial \gamma_{ij}} + \frac{\partial \tilde{\mathcal{L}}}{\partial E_j} \cdot \frac{\partial E_j}{\partial \gamma_{ij}} 
\end{equation}
In particular,
\begin{equation}
\begin{split}
\frac{\partial E_i}{\partial \gamma_{ij}} &= \frac{\partial}{\partial \gamma_{ij}} \sum_{k=1}^M E_{ik} \\
&= \frac{\partial E_{ij}}{\partial \gamma_{ij}} + \frac{\partial E_{ji}}{\partial \gamma_{ij}} \\
&= \frac{\partial}{\partial \gamma_{ij}}(w_{ij}e_i^{\gamma_{ij}} e_j^{1-\gamma_{ij}}) + \frac{\partial}{\partial \gamma_{ij}}(w_{ji}e_j^{\gamma_{ij}} e_i^{1-\gamma_{ij}})\\
&= w_{ij}e_i^{\gamma_{ij}}\frac{\partial}{\partial \gamma_{ij}}e_j^{1-\gamma_{ij}} + e_j^{1-\gamma_{ij}}\frac{\partial}{\partial \gamma_{ij}}e_i^{\gamma_{ij}} + ...\\
&= (\log e_i + \log e_j)(w_{ij}e_i^{\gamma_{ij}}e_j^{1-\gamma_{ij}} + w_{ji}e_j^{\gamma_{ij}}e_i^{1-\gamma_{ij}})\\
\end{split}
\end{equation}
As we can see, the gradient with respect to $\gamma_{ij}$ does indeed involve a natural logarithm of self-energies $e_i$ and $e_j$. Thus, self-energies must be constrained to positive values.


